-Schematron specification is built on top of XSLT and XPath
-The validation task uses an XSLT processor

XSLT implementation of Schematron, The validation is performed in two stages:
	1. The Schematron file is turned into an XSLT stylesheet by transforming it with one of XSLT stylesheet
	provided on its website (e.g. schematron-basic.xsl, schematron-message.xsl and schematron-report.xsl).
	For instance, schematron-basic.xsl permits generating simple text output
	2. The transformed Schematron file together with an XML instance document are passed to the XSLT processor
	which produces an output report (e.g text or SVRL) based on the rules and assertions in the original Schematron schema defined

It's very easy to setup a Schematron processor since it is only needed an XSLT processor together with one of
the Schematron stylesheets.

XPath implementation of Schematron:
	-faster than the XSLT approach since a transformation to an XSLT stylesheet is not needed, however
	it has less functionality, i.e XSLT-specific functions such as document() and key() functions are not
	available in XPath. For instance, contraints defined between XML documents cannot be checked using an
	XPath implementation of Schematron.

Perform a Schematron validation:

	1. Apply a set of three pre-defined XSLT scripts onto a Schematron file.
	2. After these transformations the original Schematron file is transformed into an XSLT script
	3. This XSLT script can then be applied onto XML documents for validation
	4. The output of this validation is an SVRL (Schematron Validation Report Language) document

considerations:
	1. Transforming a Schematron file into an XSLT may be very time consuming, specially for large
	definitions so it is strongly recommended to cache the resulting XSLT script

