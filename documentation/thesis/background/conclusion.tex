	In this Chapter we have explained questionnaires through a representative example that covers the spectrum of questionnaire constructs. These instruments, composed of questions and instructions, guide an interview's flow and are separated into three categories: \emph{content} to describe questions grouped into sections; \emph{routing} to decide question's sequence; and \emph{personalisation} to adapt the questionnaire structural and sequencing properties to an interviewee given their real time responses.

	Our study of different \gls{xml} schemas has shown that grammar-based schema languages are adequate to cover correctness levels such as structure and data-types, however they fail to address integrity and business constraints. In contrast, the rule-based schema formalisms and in particular \gls{sch} is best suited to specify semantics. %Accordingly, in our work we develop a mark-up language that not only maintains correctness of structure and data-types but also provides rules to validate semantics levels using only \gls{xml} schema formalisms.
