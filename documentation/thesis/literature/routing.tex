	The filter construct is well represented by \gls{sss}, \gls{qdl} and \gls{ddi} whereas it is partially described by \gls{triples} since this language is only able to represent simple logic involving only one variable. 

	In contrast, the skip logic is only defined by \gls{qdl} which considers the fact that for questionnaire designers it is difficult to express conditional statements \cite{proc:katz97}. However, as far as we know, the tool that implements this language does not offer support since it is difficult to implement conditions without having restrictions over the type of jumps allowed \cite{art:bethlehem04}. The other languages do not cover this construct either because the skips can be reversed and use filters instead \cite{art:bethke08} or since surveys designed without unstructured patterns are easy to modify, share and understand \cite{web:spencer12}. %NOT CLEAR SKIP REVERSED AND USE FILTERS INSTEAD

	The loop construct is well represented in \gls{sss} and \gls{ddi}. Despite the fact that \gls{ddi} offers three types of loops (RepeatUntil, RepeatWhile and range loop) \cite{man:thomas09}, we consider that \gls{sss} is more flexible than \gls{ddi} to define expressions since it embodies a simple functional programming style through the use of \gls{xml} tags. In spite of \gls{qdl}, it only offers support for range loop but does not consider iterations over lists (see the instruction after Q5 in Figure \ref{fig:background:survey}). Regarding \gls{triples}, it does not directly define any mechanism to iterate, although offers an interesting feature to relate data collected from hierarchical surveys (e.g. survey responses for a household survey followed by responses of the property members defined in another survey) \cite{proc:wright07}.

	The check construct is only featured in QDL and this is likely to be motivated by the need to describe multi-item constraints to check consistencies among answers to related questions \cite{proc:katz97} or alternatively because this XML language is strongly related to Blaise system \cite{proc:bethlehem00}. Regarding the computation feature, other than Triple-S, all others languages support its representation.