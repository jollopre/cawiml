	%\gls{qdl} is an \gls{xml} language developed as part of the \gls{tadeq} project. \gls{tadeq}'s aim is building a software tool for translating questionnaire specifications from different \gls{cai} systems in a human-readable format \cite{proc:bethlehem00}. This tool is able to generate two types of documentation: \emph{textual}, providing detailed information on all questionnaire constructs and \emph{graphical}, assisting in the understanding of the routing structure. The tool attempts to be helpful for different target people: \emph{questionnaire developers} who want to document their work, \emph{survey managers} who have to give formal approval to execute the survey and \emph{interviewers} who want the documentation to help them when they are collecting responses.

	%\gls{tadeq} accepts \gls{xml} instances according to \gls{qdl} which means every \gls{cai} survey system has to create its own converter from its authoring questionnaire language to \gls{qdl} (e.g. Blaise system has its own converter). \gls{qdl} is focused on design and collection stages of surveys and is able to describe filter and skip constructs indistinctly.

	\gls{qdl} is a language built for \gls{tadeq} project whose aim is at building a software tool that represents questionnaire specifications in a human-readable format \cite{proc:bethlehem00}. This tool can operate in either textual or graphical mode. It is suited to designers who want detailed information of the constructs and interviewers who need documentation to help them when they are conducting interviews. Although this project has some implementers like Blaise that has its own converter, this project has been abandoned and there is no longer support for this language.




