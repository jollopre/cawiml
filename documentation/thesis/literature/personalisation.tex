	Although \gls{sss} and \gls{ddi} offer mechanisms to describe \emph{text-fill} aspects, the other piping feature, i.e. the \emph{carry-forward} has not been considered. This construct, which permits describing operations to retrieve selected on unselected responses from previous questions as part of the responses for other questions (see Q3 from Figure \ref{fig:background:survey}), may help to better adapt surveys for each respondent. Accordingly, we consider that a specification language cannot leave out this construct. For instance, the popular SurveyMonkey \gls{cawi} system, we have observed that it implements this personalisation feature as part of its interface functionalities.

	Regarding the randomising and rotating features, these constructs are only partially covered by \gls{ddi} permitting to change the order for content constructs such as single and/or multiple question responses. However, more sophisticated patterns like selecting a specific number of responses after randomising/rotating or reordering only a subset of responses is not taken into account. For instance, the instruction above Q6a from Figure \ref{fig:background:survey} not only specifies to randomise the elements selected but also requires to iterate a maximum of four times.