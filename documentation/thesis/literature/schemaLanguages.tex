	The different \gls{xml} languages reviewed are formally defined through an \gls{xml} schema. Specifically, they use grammar-based schemas to define the vocabulary, structure and data-types expected for instances defining a questionnaire. Throughout this section, the \gls{xml} example from Listings \ref{code:background:xmlCase} will be used to discuss features supported by \gls{dtd} and \gls{xsd}.

	\gls{qdl} and \gls{triples} are defined using DTD (see Section \ref{sec:background:dtd}). This schema formalism has two weaknesses: inability to express complex structures for elements; and limited number of data-types whereby common types such as number or date are not supported.%\gls{triples} and \gls{qdl} are defined using \gls{dtd} (see Section \ref{sec:background:dtd}). This schema formalism is known for its limitations to express complex structures for elements as well as for its very limited data-types set where commonly types such as numbers or date are not supported. 
	With regards to the integrity constraints, the ID and IDREF mechanisms, offered for describing uniqueness of elements and references to valid identifiers respectively, are not robust enough for expressing semantic constraints over \gls{xml} documents. Specifically, the lexical space of an identifier is global to the entire document (e.g. the question id cannot be duplicated across different sections) and as Fan and Simeon state, this is a very strong restriction for a schema language \cite{art:fan03}. Moreover, the IDREF is not able to point to a specific key identifier (e.g. it cannot be described such that the ref attribute for a routing element links to the identifier attribute of a section). Regarding the business rules level, there is no such feature to constraint the additional semantics for \gls{xml} documents.

	\gls{sss} and \gls{ddi} use a more expressive schema language that was built to address the limitations of \gls{dtd}. Most structures are supported in \gls{xsd}. Its very rich set of data-types goes farther than simply supporting only common type such as string, boolean, decimal, integer or date to permitting the definition of any customised type through regular expressions. Regarding the integrity constraints, although a more expressive mechanism using key and key-ref through XPath expressions is provided (see Section \ref{sec:background:xsd}), not every possible relationship existing in \gls{xml} documents can be defined \cite{web:w3cxsdassertion}. Specifically, in \gls{xsd} the XPath expression for a \emph{xs:selector} can only use children and descendants of the element in which it is defined. In addition, the \emph{xs:field} restricts the XPath expressions to only select attributes \cite{proc:vandervlist06}. For instance, it is not possible to constraint the variables Q0, Q1 or Q2 such that they can point to questions defined in 'section1'. With respect to the business rules, only \gls{xsd} 1.1, which is not used neither in \gls{sss} nor \gls{ddi}, supports assertions to express additional semantics for \gls{xml} documents. However, the XPath expressions are equally limited to attributes, children and descendants of the node where the assertion has to be checked.

	The grammar-based schema languages are adequate to specify mark-up and syntax for \gls{xml} documents, however they are insufficient to express integrity constraints or business rules. Accordingly to address these issues it is best to use a rule-based schema languages such as \gls{sch} which has no restrictions on XPath expression definitions. %The grammar-based schema languages are adequate to specify mark-up and syntax for \gls{xml} documents, however they are insufficient to express any kind of integrity constraint or business rules. To better address these two other constraining levels, it is best the usage of rule-based schema languages such as \gls{sch} since there are no restrictions to define XPath expressions. 
	Therefore, if the well defined patterns from grammar-based languages are combined with the expressiveness of rule-based schemas \cite{proc:vandervlist06} \cite{web:costello15} \cite{art:lee00}, it is possible to create an \gls{xml} authoring solution that is better suited to handle the different validation stages, i.e. a language that ensures the correctness of questionnaire specifications without the necessity of relying on programming languages to validate complex semantic constraints.
