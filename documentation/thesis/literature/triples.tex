	%Triple-S is aimed to describe the key elements of surveys \cite{book:triplesxml} in order to make easier the transfer of data and meta data between survey or analysis software packages. Wright describes an use case tool which accepts data in triple-S format to take business decisions \cite{proc:wright07}.

	%This language was first published in 1994 and since then three major versions have been released. The latest one (2.0) defines two files: the \emph{definition File} which contains general information and survey variables (i.e. the meta data) and the \emph{Data File} storing data for a meta data instance. This version supports \gls{csv} format for exporting the survey data as well as the ability to specify hierarchies. The hierarchy feature results useful for questionnaires where exist a relationship parent-children (e.g. a household questionnaire contains questions at the household level and a set of questions to be repeated for each member of the household). This relationship is defined through a specification control file.

	\gls{triples} is aimed to represent the content aspects of surveys \cite{book:triplesxml} in order to make it easier to transfer data and meta data among \gls{cai} systems or any analysis software package. This language is considered the standard for representing social surveys and at least fifty registered implementers may be found on its website \cite{web:triples}. Although its focus has been to provide a comprehensive coverage of survey functionality, we have found one case study where \gls{triples} has been adapted to describe a business decision making tool \cite{proc:wright07}.