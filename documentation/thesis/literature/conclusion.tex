	We have critically analysed four \gls{xml} authoring languages to determine the coverage of questionnaire constructs. The routing features, are only fully addressed by \gls{dtd}, hardly addressed by \gls{triples} and addressed with only structured patterns by \gls{sss} and \gls{ddi}. Regarding the personalisation constructs, none of the languages explored offers support for carry-forward functionality, well represented in interfaces of \gls{cawi} solutions such as SurveyMonkey. In respect to the survey stages supported, \gls{triples} or \gls{ddi} best suit the export of survey data and meta data to other social disciplines. In contrast, \gls{sss} offers a more robust expression notation that reduces the validations needed to execute a questionnaire. However, although this prefix notation eliminates the use of parentheses, it is less efficient when compared to the postfix notation mode.

	The inability of grammar-based schema languages to address the correctness of semantics for questionnaire specifications, is evident in all the authoring solutions reviewed. Particularly, the \gls{dtd} schema formalism existing in \gls{triples} and \gls{qdl}, is very limited in its ability to express integrity constraints. Similarly, the use of \gls{xsd} through languages such as \gls{sss} and \gls{ddi} does not permit expressing any kind of relationship existing in \gls{xml} files, raising the need for using rule-based standard formalisms or general-purpose programming languages to address the semantic.

	Regarding the representation of question sequence, the hierarchical modelling adopted by \gls{qdl}, \gls{sss} and \gls{ddi} does not adequately facilitate the questionnaire logic for routing purposes. Specifically, this paradigm does not only become unsatisfactory to unify the paper specification of a questionnaire against the code produced but is also difficult to make changes when skip patterns have to be reversed, which impacts negatively over the adaptability principle.

	The study of different architectural styles for \gls{cawi} systems, shows for instance that Blaise does not support the scalability, portability and reliability properties adequately. In contrast, the stateless configuration adopted by SurveyMonkey promotes flexibility to scale systems. Respecting the simplicity, although it is sufficiently covered in both systems by separating the tasks into different services, we consider that building of web pages could be transferred to the client using the responsive \gls{spa} paradigm. Finally, from all the testing methods reviewed for \gls{cawi} system evaluation, we find that the stress testing is best to determine the capacity of a system when different parameters such as number of concurrent interviewees or think time metrics varies.