	Several authoring languages have been proposed to specify questionnaires. The two most popular solutions are \gls{cases}, that provides structured and unstructured patterns for routing and Blaise, which promotes the design of skip free logic in favour of loops, conditional branches and modules. Although these languages offer adequate construct coverage, their adoption remains restricted due the proprietary nature of the language and lack of being a standard interchange format. 

	As part of the aim to design and implement a new \gls{cawi} solution for Pexel, we have done research at two levels: one to seek responses in terms of correctness of questionnaire specifications. For that purpose, we have conducted a comparative analysis of the state-of-art \gls{xml} languages in terms of routing and personalisation constructs, notation style adopted for expressions, schema language formalism utilised, underlying flow modelling for routing or the survey stages that they support; and the other to determine the properties induced by the architectural style adopted for the most relevant \gls{cawi} solutions.
 
	The rest of this Chapter is structured as follows: Section \ref{sec:literature:xmlLanguages} introduces the \gls{xml} authoring language solutions for questionnaires before comparing them in Section \ref{sec:literature:comparativeAnalysis}. Finally, Section \ref{sec:literature:cawiSystems}, reviews the architecture styles of Blaise and SurveyMonkey \gls{cawi} systems and introduces the methodology and testing methods useful for evaluating our new \gls{cawi} solution proposed.