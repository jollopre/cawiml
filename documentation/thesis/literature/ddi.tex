	\gls{ddi} emerged in 1995 as an international project to create standardised meta data to document social science datasets \cite{art:rasmussen07}. It emerged to solve the problems of documenting datasets and introduces \gls{xml} as the exchangeable data format to be both, machine-readable and human-understandable. It has an important value for analysis and archiving since permits describing data at two levels: \emph{variable} level, which consist of describing the different variables involved on the research; and \emph{study level}, since it allow describing the population that links to the stored information. 

	In its third version was introduced the description of social surveys. Since then, the \gls{abs} has been experimenting with this exchangeable format for their design tool for questionnaires. Similarly, \gls{insee}, that uses Blaise \gls{cai} system for collecting data has shown interest on using \gls{ddi} as a standard to communicate different disciplines in the data collection field. For that purpose, a \gls{pof} was created by de Bolster to convert from Blaise language specifications to \gls{ddi} 3.1. From that experiment, it was concluded that neither \gls{ddi} instances are human readable nor compatibility between different versions is considered. For instance, when valid \gls{ddi} instances for questionnaires were verified against the schemas of the newer version at least 365 errors occurred \cite{proc:bolster13}.