	The personalisation constructs in \gls{cawiml} define the dynamic behaviour for surveys. These features, that may serve to adapt the survey for each respondent, are defined in this questionnaire language through elements such as pipe, randomising and rotating. The following subsections detail the personalisation constructs through features extracted From Figure \ref{fig:background:survey}.

	\subsubsection{Text-fill piping}
		Text-fill piping describes the behaviour of retrieving responses from previous questions as part of the text for another. For instance, in Listings \ref{code:impl:textFill} a reference to a pipe (e.g. pipe0) is described as part of the text for Q6a. This pipe, defined within a piping element points at the Inner state model and has a \gls{rpn} expression describing the current loop iterator value. Note that this value may be one of the entire set of response codes from Q2 or Q3 (e.g. A, B, C, D, E, F, G, H). For instance, if the response was A for question Q2 and B,C for question Q3, the Q6a would be repeated three times by changing the pipe value to A, B or C in its question label.
		\lstinputlisting[caption={Text-fill},label={code:impl:textFill}]{appendix/code/textFill.xml}
	\subsubsection{Carry forward piping}
		Carry-forward unlike its counterpart, is intended to capture the behaviour of populating responses for a question based on a \gls{rpn} expression that returns a list. That list commonly represents the responses selected/unselected from previous questions. For instance, Listings \ref{code:impl:carry-forward} has a multiple question (e.g. Q3) that contains a pipe reference (e.g. 'pipe0'). This pipe describes the unselected responses from Q2. The \gls{cawi} system must retrieve on real-time those unselected responses to automatically populate them as part of the responses for Q3 when the gathering of survey responses takes place. For instance, if the interviewee responded A for Q2, Q3 would be automatically populated with the responses B, C, D, E, F, G, H and Don't know, that represent those non-selected at Q2.
		\lstinputlisting[caption={Carry-forward},label={code:impl:carry-forward}]{appendix/code/carryForward.xml}
	\subsubsection{Randomising/Rotating}\label{sec:impl:Randomising}
		Randomising and rotating constructs are used to alter the data order presented to the respondent. In \gls{cawiml} these features are usually defined when a question is specified but they can also be utilised for reordering loops. There are two modes of specifying data order to both randomising and rotating: 
		\begin{itemize}
			\item \emph{All} that performs ordering of the entire set of responses and contains an attribute \emph{present} to determine the number of elements to show. For instance, the for state (see section \ref{sec:impl:for}) describes this construct to alter the iterator order and determine the maximum number of times that this loop should be repeated.
			\item \emph{Subset} which selects the elements to be randomised or rotated. Listings \ref{code:impl:randomising} defines a subset of codes (e.g. 01, 02, 03, 04, 05, 06, 07 and 08) that must be reordered randomly. Note that response 09 is not included in that set and therefore this response must appear as last choice to select for Q2. 
		\end{itemize}

		The importance of having two modes arises due to the fact that sometimes there can be responses in which their order should not be modified. For instance, it is frequent to offer responses such as don't know or not applicable at last in order to capture those interviewees that really do not know enough to have a formed opinion. For that purpose, the subset construct ensures that those responses out of the subset will remain unaltered.

		\lstinputlisting[caption={Randomising},label={code:impl:randomising}]{appendix/code/randomising.xml}