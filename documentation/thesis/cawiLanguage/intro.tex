	This Chapter presents \gls{cawiml} as an alternative authoring language to specify questionnaires only using standard \gls{xml} schema languages. Particularly, it uses a state-transition paradigm for question's sequence and is intended to facilitate the questionnaire routing logic more adequately than the popular hierarchical model. \gls{rpn}, is the expression formalism utilised for describing routing and personalisation constructs indistinctly.

	The rest of this Chapter is structured as follows: Section \ref{sec:cawiLanguage:stateTransition} introduces the state-transition routing structure. Section \ref{sec:cawiLanguage:rpn} explains the postfix notation mode as the formalism for questionnaire expressions, followed by Section \ref{sec:cawiLanguage:xmlLanguage} that explains our \gls{xml} authoring solution. Finally, \gls{xml} details for content, routing and personalisation constructs expressed in \gls{cawiml} are presented in Section \ref{sec:cawiLanguage:cawiml}.