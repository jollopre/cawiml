	In order to ensure correctness of questionnaire specification through \gls{xml}, we have presented in this Chapter \gls{cawiml} that combines \gls{xsd} together with \gls{sch} to define structure, data-types, integrity constraints and business rules using a two step validation process. This language, described through standard and platform independent schema languages, permits reduce or eliminate the necessity of using programming languages to validate survey specifications. Furthermore, as it is non-proprietary, survey agencies may benefit of sharing questionnaire specifications when surveys have to be conducted across different \gls{cawi} systems.

	\gls{cawiml} abstracts the questionnaire's routing with the state-transition paradigm. This approach integrates skip and filter constructs through an if-then-else statement and better address the conceptual gap between what survey designers want versus the code needed to produce this functionality. In particular unlike the hierarchical approach, the state-transition model avoids the necessity to reverse logics when skip patterns are present. Also, as the states perform single operations, if-then-else nested structures are non-existent thus helping to reduce coupling and improving adaptability. In addition, the unification of \gls{rpn} notation to express piping and routing constructs, not only eliminates ambiguities for complex expressions but also is faster to evaluate when compared to the prefix notation mode.

	%The combination of two \gls{xml} schema languages together with the robust \gls{rpn} formalism considered in \gls{cawiml} helps to drive the collection survey life-cycle stage. Similarly, the adoption of the state-transition to unify survey designs may benefit to create tools that document and understand questionnaire's flow for the design survey life-cycle stage.