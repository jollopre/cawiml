	\gls{cawi} is the new mode of conducting surveys through web browsers. This on-line solution extends the traditional paper questionnaire with functionality to inform the order of questions, the logic to guide question relevance and inconsistency checks to validate responses. Large scale international surveys are typically conducted by research agencies in multiple countries using \gls{cawi} systems. However, these demand for non-proprietary and platform independent questionnaire definitions that work throughout multiple survey systems.

	In this thesis, we conduct a comparative analysis at two levels: one for the different \gls{xml} authoring solutions that capture questionnaire features; and another to explore the architecture styles for the most popular \gls{cawi} solutions. The popular hierarchical model, employed to manage the questionnaire flow, is not semantically intuitive to domain experts and lacks flexibility to allow for questionnaire design refinements. An analysis of system architectures suggests that the commonly adopted multi-page paradigm to build web pages, neither reduces the server burden nor addresses the responsiveness requirements expected from survey systems.

	Accordingly to address the language shortcomings we introduce a \gls{cawiml} that uses two schema languages to validate vocabulary, structures and relationships among \gls{xml} constructs and adopts a state-transition model to manage the routing and flow of questions. \gls{cawiml} serves our \gls{rest} system to drive the design and collection stages through a single-page web build. We present our language results from testing \gls{cawiml} on a comprehensive set of real-world surveys from Pexel Research Services and use the distribution of \gls{cawiml}'s vocabulary on this sample to demonstrate its coverage of questionnaire features and effective routing support. In order to evaluate our platform, we computationally simulate both the stress test for parallel processing of requests and interviewee behaviour in terms of different user interaction response configuration levels. Results suggest that both the parallelism and variation in user behaviour can be handled within acceptable levels of usability thresholds.

	\textbf{Keywords:} survey, questionnaire, xml, mark-up, xml schema, cai, cawi, state-transition, rest, single-page.
	%We present our language results from testing \gls{cawiml} on a comprehensive set of real-world surveys from Pexel Research Services and evaluate our platform at different configuration levels of concurrency considering response time usability thresholds.