	This study has been carried out as part of the \gls{ktp} project between Robert Gordon University and Pexel Research Services \footnote{\url{http://www.pexel.co.uk/}}. This company, which is based on Glasgow, conducts telephone interviewing through its own infrastructure of units and is considered one of the largest in Europe. With the emergence of the new on-line survey mode, they are keen to consider the design and development of a new hybrid solution that permits cost reductions of training their staff to use complex interfaces.

	The first significant contribution of this work has been to design a novel \gls{xml} solution that combines two approaches for expressing constraints in \gls{xml}. The validation of correctness, conducted in a two-step process, permits reducing or eliminating the need to rely on general purpose programming languages for checking complex rules. Moreover, as the \gls{xml} language proposes the state-transition paradigm to express questionnaire's routing, not only is it better able to adapt to changes in specifications but also the use of structured patterns allows for adaptability that can support and mimic the requirements that are important to designers of questionnaires (such as social researchers).

	The second significant contribution has been the system architecture that better induces the properties of: scalability, through a stateless communication among the parts; simplicity, by transferring the web pages building to the client-side; portability, through a platform independent programming language combined with a cross platform database choice; and reliability, through loose coupling components that are easy to change under any failure.

	Secondary contributions of this research include the possibility to incorporate a searchable central repository of survey definitions according to our \gls{xml} solution. This has the potential to help researchers, who often want to know what questions can be used for a particular topic \cite{proc:corbett11}, to reuse past questions when creating new questionnaires.