	The aim of this thesis is to design and implement a new \gls{cawi} solution that provides web interfaces closer to native desktop applications. Specifically, this study is focused on the design and collection stages of surveys but also extends its functionality to cover some aspects of the management, analysis and reporting stages. The system solution has to have a clear separation of concerns, i.e. an authoring \gls{xml} solution should be utilised to create questionnaire specifications through a visual design interface and similarly these descriptions should aid to automatically drive the routing logic for survey response data collection.

	%Since this thesis is focused on the development of a web application, it is expected that the architectural design adheres to properties such as \emph{scalability}, \emph{simplicity}, \emph{portability} and \emph{reliability}. Additionally, as our aspiration is to create a commercial advantage, we expect that the combination of a novel architecture, \gls{xml} language and responsive interfaces will form important contributions to applied research in \gls{cawi} systems.

	As such, we intend to achieve our research aim with the following six objectives:

	\begin{enumerate}
		\item Conduct a comparative analysis of the state-of-art \gls{xml} language solutions that cover questionnaire definitions with a focus on the coverage of constructs and the capacity to validate correctness with standard \gls{xml} schema formalisms.
		\item Critically appraise current modelling approaches in terms of their ability to manage questionnaire flow definitions for the purposes of routing.
		\item Develop a new \gls{xml} authoring solution to better address the correctness, together with the state-transition structures necessary for routing.
		\item Analyse the architecture of different \gls{cawi} solutions in order to determine whether the necessary and sufficient properties for a \gls{cawi} system solution are induced or not. 
		\item Implement an architecture based on \gls{rest} to better handle architectural properties such as scalability, simplicity, portability or reliability.
		\item Conduct an evaluation at two levels, one for the coverage of questionnaire constructs and another to evaluate the capacity of the proposed architecture to work under different workloads.
	\end{enumerate}