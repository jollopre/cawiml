	%Context of the study
	Internet is the new medium for conducting surveys \cite{book:bethlehem12}. The accessibility to a large group of respondents, the low cost of distribution and the flexibility for the respondents to choose the right time to answer a questionnaire are factors that have made the \gls{cawi} the most popular \gls{cai} solution today. Although, benefits such as manually less demanding on interviews, improved data collection quality and fewer transcribing errors are already experienced with other \gls{cai} modes (e.g. \gls{cati}, \gls{capi}), the use of \gls{cawi} is more attractive as it eliminates the need for interviewers and consequent time restrictions to interview completion. Furthermore, it is a convenient solution requiring only a computer equipped with a browser and an Internet connection in order to conduct the survey.

	With the increasing popularity of \gls{cai} systems, the traditional questionnaires, composed of questions and instructions, have been extended to now include features that automatically decide the order of the questions, the conditions under which they have to be asked or inconsistency checks to be applied to responses.
	In order to ease the task of defining questionnaire specifications, several domain specific languages have been developed such as Blaise \cite{web:blaise15}, from the Statistics of Netherlands, or \gls{cases} by the University of California \cite{web:cases15}. These systems are aimed at capturing the requirements for question design, specifications for conditions and checks needed to route or traverse the questions space. In particular this routing requires that the inherent logical relationships between responses provided during the course of responding to a questionnaire is captured accurately. %wonder if you should have a small example figure here to explain what you mean by questionnaire logic. You can also explain what you mean by a construct.. you later use this term without explaining what it is.
	Similarly, different \gls{cawi} solutions have emerged providing visual interfaces for defining questionnaires with the purpose of reducing or eliminating the necessity of programmers for describing complex questionnaire's logic (e.g. SurveyMonkey, Qualtrics or NEBU).

	Although a \gls{cawi} system offers several advantages to its predecessors \cite{book:bethlehem12}, its choice comes at a price, since special attention must be paid to ensure all stages of functionality are managed by smart interfaces, i.e. the absence of interviewers to collect data or the task of authoring questionnaire's logic by designers, requires for richer and maintainable interfaces that quickly adapt to user's requirements. Accordingly, to address these demands, a clear separation of the presentation of the questionnaire constructs from the code that process questions and instructions is needed. %Whilst authoring languages such as Blaise or CASES already describe aspects of questionnaires appropriately.


