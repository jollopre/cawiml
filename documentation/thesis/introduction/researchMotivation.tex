	The \emph{correctness} of a questionnaire specification involves two tasks: one is checking that every construct conforms to a vocabulary, structure and content, i.e. the grammatical aspect, and the other consist of validating integrity relations and business specific constraints, i.e. the validation rules. Authoring language solutions such as Blaise and \gls{cases} sufficiently address both levels of correctness, however the problem arises when large surveys (e.g. European Community Household Panel, European Union Labour Force Survey or European Health Interview Survey among others) have to be conducted through several research agencies across different countries. Since these companies usually have different \gls{cai} solutions to support the life-cycle of survey research, the use of specifications that are only understandable by specific software solutions, raises the need for having a language solution that is \emph{non-proprietary} and \emph{platform independent}.

	Two exchangeable data formats exist today to address the requirements of reusable questionnaire specifications: one is \gls{xml}, widely used to exchange data among organisations and best known for its declarative and extensible properties; and the other is \gls{json}, increasingly used to communicate modern web applications because of its direct support by JavaScript. Intuitively, the \gls{json} choice is more attractive due to its less verbose expressive descriptions as well as its faster performance when compared to \gls{xml} \cite{art:nurseitov09}. However, it lacks support from standards such as \gls{w3c} or \gls{iso} in terms of including rules that can validate correctness of documents. Accordingly for this research we make use of XML as an authoring solution making use of its different schema solutions to express constraints. %What do you mean here by constraints? do you mean constructs?

	The \emph{routing structure} of a questionnaire is another aspect that has to be carefully considered. When the flow of a questionnaire is defined, designers or social researchers commonly use skip patterns to filter one or several questions. This enables the smooth movement from one part of the questionnaire to another by removing irrelevant questions from an interview. Despite the fact that using unstructured patterns such as GOTO to express logic in programming was stated by Djisktra \cite{art:dijkstra68}, it is widely accepted by the software community as best avoided. In contrast, social researchers, who are typically the designers of questionnaires, have no qualms about designing questionnaires with complex logic that involve multiple GOTO constructs. Therefore, a solution to find a modelling structure that describes questionnaires in a structured manner while allowing designers to express their logic without the need for programmers is needed \cite{masterthesis:madsen09}.

	In addition to the issues related to the specification of questionnaires, there exists different architectural properties that should also be considered when designing a \gls{cawi} solution:
	\begin{itemize}
		\item \emph{Scalability}, known as the capacity to work under different workloads. For instance, when an interview is carried out, keeping the status of the last operation performed for each respondent is needed. As such, this is typically solved by storing interviewee's status through a session variable making it impossible to free server resources until an interview is completed. This stateful communication impacts negatively over the infrastructure making it hard to replicate and synchronise an architecture when multiple servers can handle requests from the same respondent.
		\item \emph{Simplicity}, which is achieved through the separation of functionalities such as the user interfaces into a separated component within the server \cite{phdthesis:fielding00}. For instance, popular \gls{cawi} solutions such as SurveyMonkey use the \gls{mvc} design pattern for building multiple web pages on the server. However, with the introduction of the second generation of \gls{www}, these tasks can be moved to the client, i.e the browser, and help to reduce not only the time to complete requests but also simplify the server burden.
		\item \emph{Portability}, that permits a software solution running on different platforms. This architectural property has particular significance for web application allocation where it is important to frequently utilise services from large data centres. For that purpose, it is not desired that a \gls{cawi} solution is tied to a particular operating system or more specifically to have a database solution that is platform dependent.
		\item \emph{Reliability}, which is determined by the capacity of a system to replace a component under any failure. This property constitutes one of the most important aspects to consider since a \gls{cawi} system has to offer flexibility to allow questionnaire completion when it suits the user. 
	\end{itemize}
	
	In order to address the issues discussed above in relation to the specification of questionnaires as well as the architectural properties of \gls{cawi} systems, this thesis explores the following research questions:

	\begin{enumerate}
		\item Are the current state-of-art \gls{xml} authoring languages able to validate the correctness of questionnaire constructs using standard \gls{xml} schema languages?
		\item How can we represent the flow of questionnaires using structured patterns adapted to facilitate the questionnaire logic for routing purposes? %I removed the reference to designers, in case they ask us to evaluate it with designer feedback!
		\item Do the popular \gls{cawi} solutions consider the architectural properties of scalability, flexibility, portability and reliability adequately?
	\end{enumerate}