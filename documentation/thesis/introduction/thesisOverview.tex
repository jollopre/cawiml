	The rest of this thesis is outlined as follows: Chapter \ref{ch:background} introduces the different questionnaire features that can be used to specify surveys. We also discuss \gls{xml} as the exchangeable data format to create reusable questionnaire definitions together with an extensive comparison of the relevant \gls{xml} schema languages to validate and check correctness at grammatical and semantic levels.

	In Chapter \ref{ch:literature} we conduct a comparative analysis of the state-of-art \gls{xml} languages for questionnaire design with focus on routing and personalisation aspects, notation adopted to define questionnaire expressions, schema language utilised to define constraints as well as flow modelling for question's sequencing. Additionally, we explore different architecture styles adopted by the most relevant \gls{cawi} systems in order to determine how well they support properties such as scalability, simplicity, portability and reliability.

	Chapter \ref{ch:cawiLanguage} presents the \gls{cawiml}, a contribution of this thesis to address the validation of questionnaire specifications using \gls{xml} schema formalisms. Here we explain the state-transition paradigm adopted to model questionnaire flow and present the \gls{rpn} notation formalism to define expressions that permit filtering, computing or personalising questions and their contents. The use of \gls{cawiml}'s main features are discussed using a running example of a questionnaire.

	Chapter \ref{ch:cawiSystem} presents a \gls{rest} based architectural solution that implements the expected \gls{cawi} system properties. Different layers for communicating client and server, the business objects that address the survey life-cycle stages as well as its non-relational persistence solution to address high demands of data are presented. Additionally, the \gls{spa} paradigm adopted to build the client interfaces directly in browsers is discussed.

	Evaluation is in Chapter \ref{ch:evaluation} with results organised under two themes: coverage of questionnaire constructs by \gls{cawiml} analysed on fifteen real questionnaires; and simulated load testing of increasing numbers of concurrent users with focus on different response time thresholds.

	We conclude this thesis in Chapter \ref{ch:Conclusion} with a summary of our main contributions and desirable extensions for future work.