	The implementation of the features from fifteen real paper questionnaires into \gls{cawiml} has been successfully covered. In this experiment, we have analysed the constructs frequencies in order to determine the most popular features used at any questionnaire's category. Specifically, we have identified for content that single question is the most frequent feature utilised. Respecting the routing, the slightly differences in frequency between skip and filter constructs have led to determine that they are indistinctly used for representing conditional logic. Additionally, the checks absence at any questionnaire has confirmed its rarity across questionnaire specifications. In respect to the personalisation, we have identified that piping construct to be the most popular construct either for dynamic question labelling or for automatic population of responses.

	The stress test methodology used to evaluate our state-transition at runtime has emulated adequately the interviewee's behaviour when responding to a questionnaire. Specifically, this scenario has considered the interviewee's thinking time before responding as well as it has ensured the completion for any questionnaire initiated.

	In terms of the average response times obtained, we can conclude that the system reacts instantaneously for user levels such as 50, 100 and 150 and with seamless flow for the rest of the levels. Regarding the peak load, the system only has experienced three isolated cases at high number of users but on average the responsive limit thresholds have been met. In respect to the error rate, the use of adequate server resources together with time periods that ensure all the users do not access the system simultaneously, have permitted the \gls{cawi} system to obtain server responses free of errors. Finally, we have demonstrated that with 95\% confidence that acceptable performance is achievable with real-life questionnaires that share similar features using the proposed architecture and \gls{cawiml} solution. %Finally, through the calculation of confidence intervals at 95\% we have demonstrated that this infrastructure is able to perform similar average response times for questionnaires that share similar features.