	The client side of our \gls{cawi} solution organises the interface into three separated applications in order to address the survey life-cycle process. In these interfaces we have used \gls{html}, \gls{css} and JavaScript for structure, presentation and behaviour respectively. The following subsections detail design, collection and other interfaces that have been implemented using the \gls{spa} paradigm. 

	%The client side contains three different interfaces in order to capture the requirements expected for a \gls{cawi} solution. As these has been built separately, multiple developers can focus on different areas in order to add or modify any requirement.

	%The \emph{design} interface contains components to fully describe survey specifications for content (e.g. any type of question and its section associated), routing (e.g. describing any routing with an intuitive expression builder) and personalisation constructs (e.g. randomising and rotating aspects for response sets) according to our \gls{xml} language.

	%The \emph{collection} interface separates the question types in different templates, providing mechanisms to validate data integrity across content aspects and leaving the responsibility to decide the routing to the server. For instance the client interface offers mechanisms to check that any response provided for a question is valid according to a minimum and maximum range specified, i.e. ensuring an open integer response meets the boundaries or the number of choices selected for a multiple question is within a range.

	%The \emph{management} interface encompasses the management, analysis and reporting stages with features such as surveys active, the number of concurrent interviewees and their current location within the questionnaire, real-time charts for every type of question or mechanisms to export survey data and meta data in \gls{csv} format.