	The database solution that we have chosen to persist and retrieve survey data and meta data is a document-based \gls{nosql} approach. Through this style the database design, organised in collection of documents, offers a \emph{direct mapping} to the business objects of our \gls{cawi} solution. Additionally, as this database style weakly references data from different collections, helps to reduce the complexity of data synchronisation when information allocated in different servers has to be combined. We have carefully considered the separation of surveys into different databases, i.e. for each questionnaire provide separate archives for the data and meta data. This separation facilitates easy isolation of problem causes when dealing with the questionnaire life-cycle.

	The capacity of \gls{nosql} solutions for horizontal scaling, consisting of connecting multiple physical or virtual machines, is not only more affordable than vertical, that is focused on empowering a server with more CPU or RAM as the relational databases systems do, but also it is significantly easier to set up inducing to a more scalable database solution. %BREAK THIS SENTENCE 
	The schema-free feature permits the addition or removal of properties from data representations without the need for running migration scripts \cite{art:padhy11} and therefore provides a more flexible persistence solution. Finally, as this approach best suits data intensive applications \cite{proc:gyorodi15}, it also improves with timely handling of client requests.