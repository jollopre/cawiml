	The \gls{api} layer used to communicate client and server has been implemented through Jersey, a standard open source framework used to develop RESTful Web Services \footnote{\url{https://jersey.java.net/index.html}} in Java. Throughout this section, a Java class that captures the requirements for the collection stage of surveys (see Listings \ref{code:impl:collectionResource}) will be used to discuss the \gls{api} implemented to communicate the parts. This class defines methods (e.g. public void postToken(...)) to implement functionalities such as resume, forward and backward through an interview. Also, it specifies annotations (e.g. @POST), that permit an easy mapping between a Java class and a web resource.

	\lstinputlisting[caption={Collection resource implementation details},label={code:impl:collectionResource}]{implementation/code/collectionResource.java}

	@Path annotation describes a relative \gls{uri} path to a resource. We have defined this annotation at two levels: at the class level, that permits creating a resource identifier (e.g. \emph{/collection}); and at method level, that specifies a sub resource for a given resource (e.g. \emph{/token\{id\}} that is reachable by clients as \emph{/collection/token\{id\}}). Every request sent to the server requires a survey id for which the interview is conducted (e.g. @PathParam). @Consumes and @Produces define the exchangeable data format used to communicate client and server (e.g. MediaType.APPLICATION\_JSON) for requests and responses respectively. In addition, although it not present on the example provided, every sub resource that consumes or produces survey data and meta data requires a token header containing the interviewee's identifier. Through this header, we have achieved a stateless communication that helps for an easy scaling.

	Our \gls{rest} \gls{api} uses GET, POST and DELETE \gls{http} methods. With @GET annotation it is possible to access to sub resources such as \emph{/surveyinfo}, that retrieves the meta data of a survey like its title, description and data or \emph{/resume}, that gives flexibility for the respondents to choose the right time to continue an interview.

	The modification of sub resources is carried with @POST and @DELETE annotations. @POST, it is used for operations such as moving forward and backward through a questionnaire (e.g. postPrevious(...) or postNext(...)) and for creating a database record for a newer interviewee (e.g. postToken(...)). With respect to the @DELETE, that is used to mark a questionnaire as completed, helps to prevent the modification of survey data by interviewees.